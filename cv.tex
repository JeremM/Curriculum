\documentclass[classic]{moderncv}

% moderncv styles
\moderncvstyle{classic}

% character encoding
\usepackage[utf8]{inputenc}
\usepackage[french]{babel}
%\usepackage[latin1]{inputenc}
% personal data
\firstname{Jérémie}
\familyname{MERIGEAULT}
\title{Ingénieur Performances IT}
\address{10, rue Saint-Sabin\\ 75011 Paris, FRANCE}


\phone{06 18 77 49 91}
\email{jeremie.merigeault@gmail.com}
%\extrainfo{{\small Détenteur du permis B}}
%\renewcommand{\listsymbol}{{\fontencoding{U}\fontfamily{ding}\selectfont\tiny\symbol{'102}}}
\definecolor{see}{rgb}{0.5,0.5,0.5}

%----------------------------------------------------------------------------------
%            content
%----------------------------------------------------------------------------------

\begin{document}
\maketitle%
\makequote%

\section{Expériences Professionnelles}

\cventry{Janvier 2013 - Aujourd'hui}{ERDF, projet STM}{}{Responsable technique sur le projet d'entrepôt de données de type Big Data d'ERDF}{}{Définition et écriture des requêtes métier tactiques et analytiques sur une base de données de type cluster Hadoop et cluster Teradata. Recette technique et fonctionnelle de la première version du projet. Définition des choix d'architecture.}

\cventry{Janvier 2012 - Janvier 2013}{EDF, projet SIMM}{}{Responsable de la performance de l'application commerciale critique d'EDF (CRM, facturation \& BI) à très forte volumétrie (30 millions de clients)}{}{Pilotage opérationnel et contractuel du pôle d'expertise SAP / Oracle et des équipes technico-fonctionnelles dédiées à la performance. Réalisation des audits de performances de la production et des recettes en charge des nouvelles versions. Qualification des audits externes. Participation à la trajectoire du projet via une veille technologique, notamment sur des produits de type big data et noSQL : Exadata, Hana, Hadoop, Cassandra.}


\cventry{Février 2010 - Janvier 2013}{EDF, projet SIMM}{}{Responsable des évolutions et de la disponibilité des bases de données Oracle}{}{Coordination des équipes d'expertise Oracle. Gestion de la relation contractuelle avec l'éditeur pour le support personnalisé et le conseil. Conception et déploiement des évolutions de la base de données: Oracle 11g, Compression des données, Partitionnement, Cluster RAC. Optimisation des performances Oracle. }

\section{Compétences}
\cvitem{\textbf{Logiciels}}{Oracle Grid Infrastructure, Oracle RAC, SAP Netweaver 7 et Teradata 14.}
\cvitem{\textbf{Systèmes d'exploitation}}{Linux Debian et RedHat. Bonnes connaissances du système Unix AIX.}
\cvitem{\textbf{Langages}}{Shell, SQL en très forte volumétrie, PL/SQL, Java et \LaTeX{}.}

\section{Formation}

\cventry{2007 à 2010}{Elève Ingénieur}{}{}{INSA de ROUEN, Département Architecture des Systèmes d'Information, option Modélisation des Systèmes d'Information}{}
\cventry{2009}{Projet INSA Certifié }{}{INSA de ROUEN, Réalisation d'un projet certifié ISO9001: 2008, en équipe autonome pour le CHU de ROUEN}{Conception et implémentation de Web-Services permettant l'interrogation d'une base de données médicales. Utilisation de standards W3C ouverts}{}%Rôle d'administrateur réseau, d'administrateur de base de données Oracle et de développeur.}
\section{Divers}

\cvitem{\textbf{Langues}}{Anglais de niveau avancé, obtention de 910 sur 990 au TOEIC.\\Français, langue maternelle.}
\cvitem{\textbf{Centres d'intérêts}}{Lecture, passionné de fantastique et de science-fiction.\\Amateur de jeux de plateau et de jeux de rôles.}



\end{document}
